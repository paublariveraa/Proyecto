\documentclass{article}
\usepackage[utf8]{inputenc}

\title{Computación, conocimiento del ser lógico}
\author{paubla.rivera }
\date{March 2020}

\begin{document}

\maketitle

\section*{Introducción}
La existencia humana ha sido marcada por la búsqueda insaciable del conocimiento, aquella que nos trae de la mano a la actualidad y nos deja atravesando puertas que no sabemos con certeza si somos capaces de cruzar, y si se hará de manera correcta. En esta búsqueda del conocimiento, el ser humano descubrió una verdad que parecía infalible, las matemáticas, para algunos, lo más cercano al lenguaje de dios. ¿Pero son las matemáticas realmente algo infalible?

\section*{Desarrollo}
El mundo de la computación es un hijo de las infinitas dudas que nos afligen a diario, y que surge luego de varios siglos basado en propuestas y contradicciones de muchas mentes que parten de bases matemáticas griegas como los números naturales, aquellos a los que el matemático y lógico Leopold Kronecker se refiere con tal adoración en su frase “Dios creó los números naturales, el resto es obra de los hombres”, asumiendo con esto, que todo lo demás que conocemos es creado de forma axiomática a partir de ellos. Pero como creaciones imperfectas de dios, partimos del hecho que podemos equivocarnos, que lo construido en base a lo dado por él, puede fallar, solo ser cierto bajo ciertas condiciones. Un ejemplo de esto es la paradoja de Russell que demuestra que la teoría de conjuntos formulada por Georg Cantor y Friedrich Frege es contradictoria, pues estos manifestaban que un conjunto dado es normal o singular, es decir, se contiene así mismo o no, y no existe un punto intermedio, pero dado un conjunto C que contiene a todos los conjuntos normales, ¿este es normal o singular.? 
En busca del desarrollo constante de esta ya catalogada ciencia, se originan nuevos retos como el planteamiento de la antes mencionada teoría de conjuntos, con la que se fundamentan las matemáticas modernas. A pesar de ser una teoría aceptada en la actualidad, esta ganó en su momento varios contradictores, pues hace uso “indiscriminado” del infinito. El mismo Cantor nos habla del infinito y sus diferentes tamaños, cosa que suena de entrada un poco loco, pues si cuesta comprender que existe el infinito, hablar de tamaños es aún más complejo para algunas mentes que de por si les cuesta entender este concepto. Hilbert un matemático de gran reputación aspiraba a refundar las bases de las matemáticas para evitar las paradojas, esto se llamó “programa Hilbert” que consistía en demostrar que dichas bases (o sistemas axiomáticos) “bien definidas” tenían 3 propiedades (eran consistentes, eran finitarios y eran completos). Pero kurt Gödel en su primer teorema afirma que, bajo ciertas condiciones, ninguna teoría matemática formal es capaz de describir los números naturales y la aritmética con suficiente expresividad, es decir, que ninguna teoría es absoluta, al tratar de describir todos los sistemas y ellas recaen en contradicciones. 
La búsqueda constante del conocimiento es un agobiante para nuestra sociedad, pero también ha sido la gestora de numerosos descubrimientos e implementaciones que son útiles en cierto sentido para la supervivencia humana. Como muestra de esto, tenemos la crisis que se generó a raíz de los fundamentos de las matemáticas, que de ser un problema, pasó a ser un reto para muchos que querían encontrar la solución a todas las paradojas alrededor de este tema. Todo esto propició uno de los desarrollos más importantes de la historia, pues después de la revolución industrial, se hizo más común el uso de maquinarias para solucionar o facilitar procesos. De esta forma nacen las primeras máquinas computables como la de  Alan Turing, que demuestra que no todos los problemas matemáticos tienen solución. Esta primera máquina (la máquina universal) funcionaba en base a un algoritmo que partía de los parámetros del problema y determina si este tenía solución.


\section*{Conclusión}
En conclusión, podemos decir que las matemáticas no son una verdad absoluta, ya que con el paso del tiempo, se han demostrado restricciones en algunos de sus teoremas que se creían infalibles. Con la llegada de la computación puede decirse que se llegó al final de la llamada “crisis de las matemáticas” puesto que despejaban  las dudas de si los sistemas más difíciles y extensos eran o no resolubles. La computación en esta era moderna pasó de ser el enemigo de antaño de las matemáticas que terminó de derrumbar parte de sus bases, a ser uno de sus mejores aliados puesto que grandes recursos(en cuanto a computación se refiere) son destinados a resolver problemas matemáticos, como los santos de oro que en un tiempo pasado estaban contra atena, pero se sacrificaron por ella al destruir el muro de los lamentos en la guerra contra Hades. Podemos decir que, por lo anteriormente expuesto, son pocos los problemas matemáticos que quedan por resolver gracias a las innovaciones en la computación.

 
 
 
 
“A veces es mejor ser niños, y creer que todo es cuestión de magia.”


\section*{Referencias}
\small 

\begin{verbatim}
 
[1] K Gödel: “Los conceptos tienen una existencia objetiva” en My philo
    sophical viewpoint
    
[2] "Alan Turing: The enigma" Andrew Hodges,
    
[3] Bertrand Russell. (2020, 11 de marzo). Wikipedia, La enciclopedia libre.Fecha de consulta: 13:15, marzo 27, 2020 desde https://es.wikipedia.org/w/index.php?title=Bertrand_Russell&oldid=24187108.
    

\end{verbatim}



A veces es mejor ser niños, y creer que todo es cuestión de magia.


\end{document}
