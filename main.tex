\documentclass{article}
\usepackage[utf8]{inputenc}

\title{Conocimientos del ser lógico}
\author{paubla.rivera }
\date{March 2020}

\begin{document}

\maketitle

\section*{Introduction}
La existencia humana ha sido marcada por la búsqueda insaciable del conocimiento, aquella que nos trae de la mano a la actualidad y nos deja atravesando puertas que no sabemos con certeza si somos capaces de cruzar, y si se hará de manera correcta. Cada vez es más común la incertidumbre en lo que vemos y hacemos, como un niño que no conoce las consecuencias de sus actos, pero vive.
El mundo de la computación es un hijo de las infinitas dudas que nos afligen a diario, y que surge luego de varios siglos basado en propuestas y contradicciones de muchas mentes que parten de bases matemáticas griegas como los números naturales, aquellos a los que el matemático y lógico Leopold Kronecker se refiere con tal adoración en su frase “Dios creó los números naturales, el resto es obra de los hombres”, asumiendo con esto, que todo lo demás que conocemos es creado de forma axiomática a partir de ellos. 

\section*{Desarrollo}
En busca del desarrollo constante de esta ya catalogada ciencia, se originan nuevos retos como el planteamiento de la teoría de conjuntos por Georg Cantor, con la que se fundamentan las matemáticas modernas. A pesar de ser una teoría aceptada en la actualidad, esta ganó en su momento varios contradictores, pues hace uso “indiscriminado” del infinito. 
Hablar de lo que es infinito resulta excitante, quizás de manera inconsciente para la humanidad, ya que nos da la ilusión de que hay muchas cosas que nos quedan por descubrir, y amamos lo desconocido; eso que requiere de nuestro empeño para su comprensión. El conocimiento es un requerimiento esencial para la supervivencia (Carl Sagan), no es alocada su conjetura, vivimos a disposición de los conocimientos y en gran parte somos valorados por ello.
Pero ¿Sabemos nosotros qué es infinito?, ¿La materia?, ¿El universo?, ¿La cuarentena?
Cantor nos habla del infinito y sus diferentes tamaños, cosa que suena de entrada un poco loco, pues si cuesta comprender que existe el infinito, hablar de tamaños es aún más perturbador para algunas mentes subdesarrolladas. Sosteniendo lo mencionado por Cantor, creemos encontrar la luz al final de “tanta oscuridad”, hasta que aparece un ángel bajado del cielo y conocido como Kurt Gödel que nos llena con su revelación producida por la mismísima Atenea y contradice, o digamos que complementa lo anteriormente dicho sobre el infinito, concluyendo que este existe solo para lo desconocido.
Decir que la vida es una controversia diaria, es como redundar sobre nuestro ser, que vive atormentado por descubrir la verdad y lo correcto, llenamos nuestra existencia de numerosas teorías que giran entorno a un mismo tema del cual probablemente nunca conoceremos a profundidad su razón.
Cabe mencionar que esta inquietud que nos caracteriza no solo es un agobiante para nuestra sociedad, también ha sido la gestora de numerosos descubrimientos e implementaciones que son útiles en cierto sentido para la supervivencia humana. 
Como muestra de esto, tenemos la crisis que se generó a raíz de los fundamentos de las matemáticas, que, de ser un problema, pasó a ser un reto para muchos que querían encontrar la solución a todas las paradojas alrededor de este tema. Todo esto propició uno de los desarrollos más importantes de la historia, pues después de la revolución industrial se hizo más común el uso de maquinarias para solucionar o facilitar procesos, y de esta forma nacen las primeras máquinas computables con las que Alan Turing demuestra que no todos los problemas tienen solución.

\section*{Conclusion}
Esta conclusión de Turing nos deja con computadoras y con un pedacito de incertidumbre para los curiosos e insaciables que estarán dispuestos a continuar en la búsqueda ardua del conocimiento, que, si bien enriquece en ciertos aspectos, hoy en día, sabemos que se convierte en un arma de doble filo, que satisface necesidades y genera otras.

A veces es mejor ser niños, y creer que todo es cuestión de magia.

\section*{Referencias}
\small 

\begin{verbatim}
 
[1] K Gödel: “Los conceptos tienen una existencia objetiva” en My philo
    sophical viewpoint
    
[2] "Alan Turing: The enigma" Andrew Hodges,
    
[3] Bertrand Russell. (2020, 11 de marzo). Wikipedia, La enciclopedia libre.Fecha de consulta: 13:15, marzo 27, 2020 desde https://es.wikipedia.org/w/index.php?title=Bertrand_Russell&oldid=24187108.
    

\end{verbatim}



A veces es mejor ser niños, y creer que todo es cuestión de magia.


\end{document}
